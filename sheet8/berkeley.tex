\documentclass{article}

\usepackage{geometry}
 \geometry{
 a4paper,
 total={170mm,250mm},
 left=20mm,
 top=20mm,
 }
\usepackage{fancyhdr}
\pagestyle{fancy}
\addtolength{\headheight}{\baselineskip}
\usepackage{datetime}

\usepackage{tikz}
\usetikzlibrary{shapes,arrows}
\tikzstyle{decision} = [diamond, draw, fill=blue!20, 
    text width=4.5em, text badly centered, node distance=2.2cm, inner sep=0pt]
\tikzstyle{block} = [rectangle, draw, fill=blue!20, 
    text width=18em, text centered, rounded corners, minimum height=2.5em]
\tikzstyle{line} = [draw, -latex']

\usepackage{wrapfig}
\usepackage{amsmath}
\usepackage{mathtools}
\usepackage{enumitem}
\usepackage{stmaryrd}

\def\theauthor{Alexander Bigalke \\ Robin Rombach \\ Arthur Heimbrecht}
\def\theassignment{exercise 8}
\def\theduedate{\formatdate{18}{07}{2018}}

\renewcommand{\headrulewidth}{0.4pt}
\renewcommand{\footrulewidth}{0.4pt}

\fancyfoot[L]{Class: \textit{Advanced Machine Learning}}
\fancyfoot[R]{\today}
\fancyfoot[C]{File: \\ \jobname.pdf}

\fancyhead[L]{Assignment: \textit{\theassignment} \\ Due: \theduedate}
\fancyhead[R]{Name: \theauthor}

\usepackage{graphicx}

\begin{document}

\section*{Berkeley Admission}
First we want to complete the data. We need the probabilities $p(F|G), p(A|F,G)$ and $p(G)$.\\

\begin{tabular}{| l | c | c | c | c |}
  \hline
  & \multicolumn{2}{| c |}{men} & \multicolumn{2}{| c |}{women} \\
  field & $p(F|G)$ & $p(A|F,G)$	& $p(F|G)$ & $p(A|F,G)$ \\ 
  \hline
  A & 0.62 & 0.32 & 0.82 & 0.06 \\
  B & 0.63 & 0.22 & 0.68 & 0.01 \\
  C & 0.37 & 0.13 & 0.34 & 0.32 \\
  D & 0.33 & 0.16 & 0.35 & 0.20 \\
  E & 0.28 & 0.07 & 0.24 & 0.21 \\
  F & 0.06 & 0.10 & 0.07 & 0.19 \\
  \hline
  & \multicolumn{2}{| c |}{0.585} & \multicolumn{2}{| c |}{0.415} \\
  \hline
\end{tabular}
\subsection*{Total Causal Effect}
\begin{enumerate}
	\item we can straight forward calculate the conditional probability $p_1(A = \text{true} | do(G))$: \\
	\begin{equation} p_1(A  = \text{true} | do(G)) = p_1(A = \text{true} | G) = \sum_{F} p(F|G) p(A  = \text{true} | F, G) \end{equation}

	\item Now we can calculate
		\begin{align*}
		p_1(A = \text{true} | do(G = \text{male})) & =  0.32 \cdot 0.62 + 0.22 \cdot 0.63 + ... = 0.46 \\
		p_1(A = \text{true} | do(G = \text{female})) & =  0.0.06 \cdot 0.82 + 0.01 \cdot 0.68 + ... = 0.30
		\end{align*}
	This indicates that women might be discriminated.
\end{enumerate}
\subsection*{Direct Causal Effect}
\begin{enumerate}
	\item we can calculate $p_2(A = \text{true} | G)$: \\
	\begin{align*}
		p_2(A = \text{true} | G) & = p_1(A = \text{true} | G, \text{cut}(G \rightarrow A)) \\
					& = \sum_{F} p_1(F|G)  p_1(A = \text{true} | F, G, \text{cut}(G \rightarrow A)) \\
					& = \sum_{F}\sum_{\tilde{G}} p_1(F|G)  p_1(A = \text{true} | F, \tilde{G}) p(\tilde{G}) \\
					& = \sum_{\tilde{G}} p(\tilde{G}) \sum_{F} p_1(F|G)  p_1(A = \text{true} | F, \tilde{G}) \\
					& = p(\tilde{G} = \text{male})\sum_{F} p_1(F|G)  p_1(A = \text{true} | F, \tilde{G} = \text{male}) \\
					& + p(\tilde{G} = \text{female})\sum_{F} p_1(F|G)  p_1(A = \text{true} | F, \tilde{G} = \text{female})\\
	\end{align*}

	\item Now we can calculate
		\begin{align*}
		p_2(A = \text{true} | G = \text{male}) & =  0.585 \cdot p_1(A = \text{true} | do(G = \text{male})) + 0.415 \cdot (0.82 \cdot 0.32 + ... )\\
							&  = 0.585 \cdot 0.46 + 0.415 \cdot 0.54 = 0.49 \\
		p_2(A = \text{true} | G = \text{female}) & =  0.415 \cdot p_1(A = \text{true} | do(G = \text{female})) + 0.585 \cdot (0.06 \cdot 0.62 + ... )\\
							&  = 0.415 \cdot 0.30 + 0.585 \cdot 0.30 = 0.30 \\
		\end{align*}
	The results yield the same conditional probability for women in both models indicating that there is no discrimination in that respect. But for men the direct causal effect yields a higher conditional probability thus indicating that there might be little discrimination against male applicants.
\end{enumerate}
\end{document}


